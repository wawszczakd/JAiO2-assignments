\documentclass[12pt]{article}
\usepackage[utf8]{inputenc}
\usepackage[T1]{fontenc}
\usepackage{indentfirst}
\usepackage{amsmath}
\usepackage{amssymb}
\usepackage{natbib}
\usepackage{graphicx}
\usepackage{float}
\usepackage[a4paper, margin = 2 cm]{geometry}
\usepackage{fancyhdr}
\usepackage{wrapfig}
\usepackage{hyperref}
\usepackage{mathtools}

\title{Languages, Automata and Computation II assignment 1}
\author{Dominik Wawszczak}
\date{2024-11-12}

\begin{document}
	\setlength{\parindent}{0 cm}
	
	Dominik Wawszczak \hfill Languages, Automata and Computation II
	
	Student ID Number: 440014 \hfill Assignment 1
	
	Group Number: 1
	
	\bigskip
	\hrule
	\bigskip
	
	\textbf{Problem 1}
	
	\medskip
	
	We define an \(\omega\)-word \(u\) as \textit{universal} if, and only if,
	for any \(i \in \mathbb{N}\), there exists a \(j > i\) such that \(u[i] \neq
	u[j]\). Intuitively, this means that \(u\) contains an infinite subsequence
	of the form \((ab)^{\infty}\). Note that if \(u \in \Sigma^{\omega}\) is
	universal, then for any \(v \in \Sigma^{\omega}\), it holds that \(v
	\sqsubseteq u\). This is because we can remove finitely or infinitely many
	letters from \(u\) to obtain \((ab)^{\infty}\), and for each \(i\)-th pair
	of consecutive \(ab\) occurrences, we can remove either \(a\) or \(b\),
	depending on the value of \(v[i]\).
	
	\medskip
	
	If an \(\omega\)-word \(u\) is not universal, then either \(\#_{a}(u)\) or
	\(\#_{b}(u)\) is finite. In the first case, \(u\) is of the form
	\(vb^{\infty}\), and we will refer to such an \(\omega\)-word as
	\textit{\(a\)-long}. In the second case, \(u\) is of the form
	\(va^{\infty}\), and we will call it \textit{\(b\)-long}. In both cases,
	\(v\) is a word of finite length.
	
	\medskip
	
	Let \(u_{1}, u_{2}, \ldots\) be any infinite sequence of \(\omega\)-words.
	We will prove that there exist indices \(i < j\) such that \(u_{i}
	\sqsubseteq u_{j}\). If there are at least two universal \(\omega\)-words in
	this sequence, the proof is straightforward since \(u_{1}\) can embed into
	the second universal word in the sequence. Otherwise, either there are
	infinitely many \(a\)-long words in the sequence, or there are infinitely
	many \(b\)-long words. Without loss of generality, assume there are
	infinitely many \(a\)-long words in the sequence. Let \(k_{1}, k_{2},
	\ldots\) be the indices of the \(a\)-long words. Denote \(u_{k_{l}} =
	u_{k_{l}}'a^{\infty}\), where \(u_{k_{l}}'\) is finite. By Higman's lemma,
	there exist indices \(i < j\) such that \(u_{k_{i}}'\) is a substring of
	\(u_{k_{j}}'\), concluding the proof as \(u_{k_{i}} \sqsubseteq u_{k_{j}}\).
	
	\medskip
	
	Now, we will show that in the variant where only finitely many letters can
	be removed, the resulting relation is not a well-quasi order. Consider the
	infinite sequence of \(\omega\)-words \(u_{1}, u_{2}, \ldots\) such that
	\[ u_{i} \ = \ a^{i} b a^{i + 1} b a^{i + 2} b \ldots \]
	
	Clearly, for any \(i \in \mathbb{N}^{+}\), we have \(u_{i} \sqsupseteq
	u_{i + 1}\) because the first \(i + 1\) letters of \(u_{i}\) can be removed
	to obtain \(u_{i + 1}\).
	
	\medskip
	
	Suppose, for the sake of contradiction, that there exists \(i \in
	\mathbb{N}^{+}\) such that \(u_{i} \sqsubseteq u_{i + 1}\). Let \(k\) be the
	smallest number such that a non expandable block of the form \(a^{k} b\)
	remains intact, after removing finitely many letters from \(u_{i + 1}\) to
	obtain \(u_{i}\). The number of letters \(b\) in \(u_{i + 1}\) before this
	block equals \(k - i - 1\). However, the number of letters \(b\) in
	\(u_{i}\) before this block equals \(k - i > k - i - 1\), which is a
	contradiction, as letters can only be removed.
	
	Suppose, for the sake of contradiction, that there exists \(i \in
	\mathbb{N}^{+}\) such that \(u_{i} \sqsubseteq u_{i + 1}\). Let \(k\) be the
	smallest number such that a non-expandable block of the form \(a^{k} b\)
	remains intact after removing finitely many letters from \(u_{i + 1}\) to
	obtain \(u_{i}\). The number of \(b\) letters in \(u_{i + 1}\) before this
	block is \(k - i - 1\). However, the number of \(b\) letters in \(u_{i}\)
	before this block is \(k - i\), which is greater than \(k - i - 1\),
	contradicting the fact that letters can only be removed.
	
	\medskip
	
	From the above, we conclude that \(u_{i} \sqsupset u_{i + 1}\) for any \(i
	\in \mathbb{N}^{+}\). Therefore, \(u_{1}, u_{2}, \ldots\) forms an infinite
	sequence of strictly decreasing elements, which shows that the relation is
	not a well-quasi order.
\end{document}
