\documentclass[12pt]{article}
\usepackage[utf8]{inputenc}
\usepackage[T1]{fontenc}
\usepackage{indentfirst}
\usepackage{amsmath}
\usepackage{amssymb}
\usepackage{natbib}
\usepackage{graphicx}
\usepackage{float}
\usepackage[a4paper, margin = 2 cm]{geometry}
\usepackage{fancyhdr}
\usepackage{wrapfig}
\usepackage{hyperref}
\usepackage{mathtools}

\title{Languages, Automata and Computation II assignment 1}
\author{Dominik Wawszczak}
\date{2024-11-12}

\begin{document}
	\setlength{\parindent}{0 cm}
	
	Dominik Wawszczak \hfill Languages, Automata and Computation II
	
	Student ID Number: 440014 \hfill Assignment 1
	
	Group Number: 1
	
	\bigskip
	\hrule
	\bigskip
	
	\textbf{Problem 1}
	
	\medskip
	
	We define an \(\omega\)-word \(u\) as \textit{universal} if, and only if,
	for any \(i \in \mathbb{N}\), there exists a \(j > i\) such that \(u[i] \neq
	u[j]\). Intuitively, this means that \(u\) contains an infinite subsequence
	of the form \((ab)^{\infty}\). Note that if \(u \in \Sigma^{\omega}\) is
	universal, then for any \(v \in \Sigma^{\omega}\), it holds that \(v
	\sqsubseteq u\). This is because we can remove finitely or infinitely many
	letters from \(u\) to obtain \((ab)^{\infty}\), and for each \(i\)-th pair
	of consecutive \(ab\) occurrences, we can remove either \(a\) or \(b\),
	depending on the value of \(v[i]\).
	
	\medskip
	
	If an \(\omega\)-word \(u\) is not universal, then either \(\#_{a}(u)\) or
	\(\#_{b}(u)\) is finite. In the first case, \(u\) is of the form
	\(vb^{\infty}\), and we will refer to such an \(\omega\)-word as
	\textit{\(a\)-long}. In the second case, \(u\) is of the form
	\(va^{\infty}\), and we will call it \textit{\(b\)-long}. In both cases,
	\(v\) is a word of finite length.
	
	\medskip
	
	Let \(u_{1}, u_{2}, \ldots\) be any infinite sequence of \(\omega\)-words.
	We will prove that there exist indices \(i < j\) such that \(u_{i}
	\sqsubseteq u_{j}\). If there are at least two universal \(\omega\)-words in
	this sequence, the proof is straightforward since \(u_{1}\) can embed into
	the second universal word in the sequence. Otherwise, either there are
	infinitely many \(a\)-long words in the sequence, or there are infinitely
	many \(b\)-long words. Without loss of generality, assume there are
	infinitely many \(a\)-long words in the sequence. Let \(k_{1}, k_{2},
	\ldots\) be the indices of the \(a\)-long words. Denote \(u_{k_{l}} =
	u_{k_{l}}'a^{\infty}\), where \(u_{k_{l}}'\) is finite. By Higman's lemma,
	there exist indices \(i < j\) such that \(u_{k_{i}}'\) is a substring of
	\(u_{k_{j}}'\), concluding the proof as \(u_{k_{i}} \sqsubseteq u_{k_{j}}\).
	
	\medskip
	
	Now, we will show that in the variant where only finitely many letters can
	be removed, the resulting relation is not a well-quasi-order. Consider the
	infinite sequence of \(\omega\)-words \(u_{1}, u_{2}, \ldots\) such that
	\[ u_{i} \ = \ a^{i} b a^{i + 1} b a^{i + 2} b \ldots \]
	
	Clearly, for any \(i \in \mathbb{N}^{+}\), we have \(u_{i} \sqsupseteq
	u_{i + 1}\) because the first \(i + 1\) letters of \(u_{i}\) can be removed
	to obtain \(u_{i + 1}\).
	
	\medskip
	
	Suppose, for the sake of contradiction, that there exists \(i \in
	\mathbb{N}^{+}\) such that \(u_{i} \sqsubseteq u_{i + 1}\). Let \(k\) be the
	smallest number such that a non expandable block of the form \(a^{k} b\)
	remains intact, after removing finitely many letters from \(u_{i + 1}\) to
	obtain \(u_{i}\). The number of letters \(b\) in \(u_{i + 1}\) before this
	block equals \(k - i - 1\). However, the number of letters \(b\) in
	\(u_{i}\) before this block equals \(k - i > k - i - 1\), which is a
	contradiction, as letters can only be removed.
	
	\medskip
	
	Suppose, for the sake of contradiction, that there exists \(i \in
	\mathbb{N}^{+}\) such that \(u_{i} \sqsubseteq u_{i + 1}\). Let \(k\) be the
	smallest number such that a non-expandable block of the form \(a^{k} b\)
	remains intact after removing finitely many letters from \(u_{i + 1}\) to
	obtain \(u_{i}\). The number of \(b\) letters in \(u_{i + 1}\) before this
	block is \(k - i - 1\). However, the number of \(b\) letters in \(u_{i}\)
	before this block is \(k - i\), which is greater than \(k - i - 1\),
	contradicting the fact that letters can only be removed.
	
	\medskip
	
	From the above, we conclude that \(u_{i} \sqsupset u_{i + 1}\) for any \(i
	\in \mathbb{N}^{+}\). Therefore, \(u_{1}, u_{2}, \ldots\) forms an infinite
	sequence of strictly decreasing elements, which shows that the relation is
	not a well-quasi-order.
	
	\newpage
	
	\textbf{Problem 2}
	
	\medskip
	
	For any set \(A \subseteq \{1, \ldots, d\}\), define a function \(f_{A} :
	\mathbb{N} \to \mathbb{N}^{d}\) as follows:
	\[ f_{A}(n) \ = \ ([1 \in A] \cdot n, \ldots, [d \in A] \cdot n) \text{,} \]
	where \([P]\) denotes the Iverson bracket, i.e., \([P] = 1\) if \(P\) is
	true, and \([P] = 0\) otherwise. A set \(A\) is called \textit{good} if and
	only if \(f_{A}(n) \in X\) for every \(n \in \mathbb{N}\). Note that
	\(\emptyset\) is good unless \(X\) is empty, which is a trivial case.
	
	\medskip
	
	For any good set \(A \subseteq \{1, \ldots, d\}\), let \(g(A)\) denote the
	set of tuples \((a_{1}, \ldots, a_{d})\) satisfying the following conditions:
	\begin{enumerate}
		\item \(a_{i} = 0\), if \(i \in A\);
		\item \(\underset{n \in \mathbb{N}}{\forall} \ (a_{1}, \ldots, a_{d}) +
		      f_{A}(n) \in X\);
		\item \(\underset{n_{0} \in \mathbb{N}}{\exists} \ \underset{n \geqslant
		      n_{0}}{\forall} \ \underset{i \in \{1, \ldots, d\} \setminus A}
		      {\forall} \ (a_{1}, \ldots, a_{i - 1}, a_{i} + 1, a_{i + 1},
		      \ldots, a_{d}) + f_{A}(n) \notin X\),
	\end{enumerate}
	where \((a_{1}, \ldots, a_{d}) + (b_{1}, \ldots, b_{d}) = (a_{1} + b_{1},
	\ldots, a_{d} + b_{d})\), for any tuples \((a_{1}, \ldots, a_{d})\) and
	\((b_{1}, \ldots, b_{d})\) in \(\mathbb{N}^{d}\).
	
	\medskip
	
	Suppose, for the sake of contradiction, that \(g(A)\) is infinite. Let
	\[ g(A) \ = \ \big\{ \big( a_{1}^{1}, \ldots, a_{d}^{1} \big), \big(
	a_{1}^{2}, \ldots, a_{d}^{2} \big), \ldots \big\} \]
	By Dickson's lemma, the structure \(\big( \mathbb{N}^{d}, \leqslant \!\!
	\big)\) is a well-quasi-order.  Therefore, there exist indices \(i, j \in
	\mathbb{N}\) such that \(i < j\) and \(\big( a_{1}^{i}, \ldots, a_{d}^{i}
	\big) \leqslant \big( a_{1}^{j}, \ldots, a_{d}^{j} \big)\), contradicting
	the third condition for tuples in \(g(A)\).
	
	\medskip
	
	Define \(\text{down}(g(A))\) as the set of all tuples \((b_{1}, \ldots,
	b_{d}) \in \mathbb{N}^{d}\) such that there exists a tuple \((a_{1}, \ldots,
	a_{d}) \in g(A)\) with \((a_{1}, \ldots, a_{d}) \geqslant (b_{1}, \ldots,
	b_{d})\). If \(g(A)\) is empty, let \(\text{down}(g(A))\) be the singleton
	containing the zero tuple.
	
	\medskip
	
	The set \(\text{down}(g(A))\) is finite, with the following upper bound on
	its size:
	\[ |\text{down}(g(A))| \ \leqslant \ \sum\limits_{(a_{1}, \ldots, a_{d}) \in
	g(A)} \ \prod\limits_{i = 1}^{d} (a_{i} + 1) \text{.} \]
	Using this, define a semilinear set
	\[ Y \ = \ \bigcup_{A \subseteq \{1, \ldots, d\} \ \wedge \ A \
	\text{is good}} \left( \text{down}(g(A)) + \left( \bigcup_{i \in A}
	([1 = i], \ldots, [d = i]) \right)^{\ast} \right) \text{.} \]
	By definition of \(g(A)\), we have \(Y \subseteq X\). Our goal is to show
	that \(Y = X\).
	
	\medskip
	
	Suppose, for the sake of contradiction, that there exists a tuple \((a_{1},
	\ldots, a_{d}) \in X \setminus Y\). We will construct a tuple \((b_{1},
	\ldots, b_{d}) \in X\) such that \((b_{1}, \ldots, b_{d}) \geqslant (a_{1},
	\ldots, a_{d})\). Since \(Y\) is downward closed, this will imply \((b_{1},
	\ldots, b_{d}) \in X \setminus Y\).
	
	\medskip
	
	Proceed through indices \(i = 1, \ldots, d\), choosing \(b_{i}\) iteratively
	while maintaining a set \(A\), initially empty. Assume \(b_{1}, \ldots,
	b_{i - 1}\) have been chosen. For all \(j \in \{1, \ldots, i - 1\}\), define
	\(b_{j}' = b_{j}\) if \(j \in A\), and \(b_{j}' = 0\) otherwise. If there
	exists a number \(c_{i} \in \mathbb{N}\) such that
	\begin{itemize}
		\item \(\underset{n \in \mathbb{N}}{\forall} \ (b_{1}', \ldots,
		      b_{i - 1}', c_{i}, a_{i + 1}, \ldots, a_{d}) + f_{A}(n) \in X\),
		      and
		\item \(\underset{n_{0} \in \mathbb{N}}{\exists} \ \underset{n \geqslant
		      n_{0}}{\forall} \ (b_{1}', \ldots, b_{i - 1}', c_{i} + 1,
		      a_{i + 1}, \ldots, a_{d}) + f_{A}(n) \notin X\),
	\end{itemize}
	set \(b_{i} = c_{i}\). Otherwise, let \(b_{i} = a_{i}\) and add \(i\) to
	\(A\). Note that \(b_{i} \geqslant a_{i}\).
	
	\medskip
	
	By construction, \((b_{1}, \ldots, b_{d}) \in X\), and \(A\) is good.
	Furthermore,
	\[ \underset{n_{0} \in \mathbb{N}}{\exists} \ \underset{n \geqslant n_{0}}
	{\forall} \ \underset{i \in \{1, \ldots, d\} \setminus A} {\forall} \
	(b_{1}, \ldots, b_{i - 1}, b_{i} + 1, b_{i + 1}, \ldots, b_{d}) + f_{A}(n)
	\notin X \text{.} \]
	Thus,
	\[ (b_{1}, \ldots, b_{d}) \ \in \ \text{down}(g(A)) + \left(
	\bigcup_{i \in A} ([1 = i], \ldots, [d = i]) \right)^{\ast} \text{.} \]
	This contradiction completes the proof that \(X\) is a semilinear set.
\end{document}
