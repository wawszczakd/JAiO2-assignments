\documentclass[12pt]{article}
\usepackage[utf8]{inputenc}
\usepackage[T1]{fontenc}
\usepackage{indentfirst}
\usepackage{amsmath}
\usepackage{amssymb}
\usepackage{natbib}
\usepackage{graphicx}
\usepackage{float}
\usepackage[a4paper, margin = 2 cm]{geometry}
\usepackage{fancyhdr}
\usepackage{wrapfig}
\usepackage{hyperref}
\usepackage{mathtools}
\usepackage{algorithm}
\usepackage{algpseudocode}

\title{Languages, Automata and Computation II assignment 1}
\author{Dominik Wawszczak}
\date{2024-11-12}

\begin{document}
	\setlength{\parindent}{0 cm}
	
	Dominik Wawszczak \hfill Languages, Automata and Computation II
	
	Student ID Number: 440014 \hfill Assignment 1
	
	Group Number: 1
	
	\bigskip
	\hrule
	\bigskip
	
	\textbf{Problem 1}
	
	\medskip
	
	We define an \(\omega\)-word \(u\) as \textit{universal} if, and only if,
	for any \(i \in \mathbb{N}\), there exists a \(j > i\) such that \(u[i] \neq
	u[j]\). Intuitively, this means that \(u\) contains an infinite subsequence
	of the form \((ab)^{\infty}\). Note that if \(u \in \Sigma^{\omega}\) is
	universal, then for any \(v \in \Sigma^{\omega}\), it holds that \(v
	\sqsubseteq u\). This is because we can remove finitely or infinitely many
	letters from \(u\) to obtain \((ab)^{\infty}\), and for each \(i\)-th pair
	of consecutive \(ab\) occurrences, we can remove either \(a\) or \(b\),
	depending on the value of \(v[i]\).
	
	\medskip
	
	If an \(\omega\)-word \(u\) is not universal, then either \(\#_{a}(u)\) or
	\(\#_{b}(u)\) is finite. In the first case, \(u\) is of the form
	\(vb^{\infty}\), and we will refer to such an \(\omega\)-word as
	\textit{\(a\)-long}. In the second case, \(u\) is of the form
	\(va^{\infty}\), and we will call it \textit{\(b\)-long}. In both cases,
	\(v\) is a word of finite length.
	
	\medskip
	
	Let \(u_{1}, u_{2}, \ldots\) be any infinite sequence of \(\omega\)-words.
	We will prove that there exist indices \(i < j\) such that \(u_{i}
	\sqsubseteq u_{j}\). If there are at least two universal \(\omega\)-words in
	this sequence, the proof is straightforward since \(u_{1}\) can embed into
	the second universal word in the sequence. Otherwise, either there are
	infinitely many \(a\)-long words in the sequence, or there are infinitely
	many \(b\)-long words. Without loss of generality, assume there are
	infinitely many \(a\)-long words in the sequence. Let \(k_{1}, k_{2},
	\ldots\) be the indices of the \(a\)-long words. Denote \(u_{k_{l}} =
	v_{k_{l}}a^{\infty}\), where \(v_{k_{l}}\) is finite. By Higman's lemma,
	there exist indices \(i < j\) such that \(v_{k_{i}}\) is a substring of
	\(v_{k_{j}}\), concluding the proof as \(u_{k_{i}} \sqsubseteq u_{k_{j}}\).
	
	\medskip
	
	Now, we will show that in the variant where only finitely many letters can
	be removed, the resulting relation is not a well-quasi-order. Consider the
	infinite sequence of \(\omega\)-words \(u_{1}, u_{2}, \ldots\) such that
	\[ u_{i} \ = \ a^{i} b a^{i + 1} b a^{i + 2} b \ldots \]
	
	Clearly, for any \(i \in \mathbb{N}^{+}\), we have \(u_{i} \sqsupseteq
	u_{i + 1}\) because the first \(i + 1\) letters of \(u_{i}\) can be removed
	to obtain \(u_{i + 1}\).
	
	\medskip
	
	Suppose, for the sake of contradiction, that there exists \(i \in
	\mathbb{N}^{+}\) such that \(u_{i} \sqsubseteq u_{i + 1}\). Let \(k\) be the
	smallest number such that a non expandable block of the form \(a^{k} b\)
	remains intact, after removing finitely many letters from \(u_{i + 1}\) to
	obtain \(u_{i}\). The number of letters \(b\) in \(u_{i + 1}\) before this
	block equals \(k - i - 1\). However, the number of letters \(b\) in
	\(u_{i}\) before this block equals \(k - i > k - i - 1\), which is a
	contradiction, as letters can only be removed.
	
	\medskip
	
	From the above, we conclude that \(u_{i} \sqsupset u_{i + 1}\) for any \(i
	\in \mathbb{N}^{+}\). Therefore, \(u_{1}, u_{2}, \ldots\) forms an infinite
	sequence of strictly decreasing elements, which shows that the relation is
	not a well-quasi-order.
	
	\bigskip
	
	\textbf{Problem 2}
	
	\medskip
	
	Consider the structure \(\big( (\mathbb{N} \cup \{\omega\})^{d}, \leqslant
	\! \big)\), where \((x_{1}, \ldots, x_{d}) \leqslant (y_{1}, \ldots,
	y_{d})\) if and only if, for each \(i \in \{1, \ldots, d\}\), one of the
	following holds:
	\[ x_{i} = y_{i} = \omega \qquad \text{or} \qquad x_{i} \in \mathbb{N} \
	\wedge \ y_{i} = \omega \qquad \text{or} \qquad x_{i}, y_{i} \in \mathbb{N}
	\ \wedge \ x_{i} \leqslant y_{i} \text{.} \]
	By Dickson's lemma, this structure is a well-quasi-order because
	\((\mathbb{N} \cup \{\omega\}, \leqslant)\) is itself a well-quasi-order.
	
	\medskip
	
	For any tuple \((x_{1}, \ldots, x_{d}) \in (\mathbb{N} \cup
	\{\omega\})^{d}\), we say \((x_{1}, \ldots, x_{d}) \in X\) if and only if,
	for every \(n \in \mathbb{N}\), the following condition holds:
	\[ ([x_{1} = \omega] \cdot n + [x_{1} \neq \omega] \cdot x_{1}, \ldots,
	[x_{d} = \omega] \cdot n + [x_{d} \neq \omega] \cdot x_{1}) \ \in \ X
	\text{,} \]
	where \([P]\) denotes the Iverson bracket, i.e., \([P] = 1\) if \(P\) is
	true, and \([P] = 0\) otherwise.
	
	\medskip
	
	A subset \(A \subseteq \{1, \ldots, d\}\) is called \textit{good} if and
	only if \(([1 \in A] \cdot \omega, \ldots, [d \in A] \cdot \omega) \in X\).
	For any good set \(A\), define \(g(A)\) as the set of tuples \((x_{1},
	\ldots, x_{d}) \in (\mathbb{N} \cup \{\omega\})^{d}\) satisfying:
	\begin{enumerate}
		\item \(x_{i} = \omega\), for all \(i \in A\);
		\item \((x_{1}, \ldots, x_{d}) \in X\);
		\item \((x_{1}, \ldots, x_{i - 1}, x_{i} + 1, x_{i + 1}, \ldots, x_{d})
		      \notin X\), for all \(i \in \{1, \ldots, d\} \setminus A\).
	\end{enumerate}
	
	Suppose, for the sake of contradiction, that \(g(A)\) is infinite. Let
	\[ g(A) \ = \ \big\{ \big( x_{1}^{1}, \ldots, x_{d}^{1} \big), \big(
	x_{1}^{2}, \ldots, x_{d}^{2} \big), \ldots \big\} \text{.} \]
	Since \(\big( (\mathbb{N} \cup \{\omega\})^{d}, \leqslant \! \big)\) is a
	well-quasi-order, there exist indices \(i, j \in \mathbb{N}^{+}\) such that
	\(i < j\) and \(\big( x_{1}^{i}, \ldots, x_{d}^{i} \big) \leqslant \big(
	x_{1}^{j}, \ldots, x_{d}^{j} \big)\), contradicting the third condition for
	tuples in \(g(A)\).
	
	\medskip
	
	Define \(\text{down}(g(A))\) as the set of all tuples \((y_{1}, \ldots,
	y_{d}) \in \mathbb{N}^{d}\) for which there exists a tuple \((x_{1}, \ldots,
	x_{d}) \in g(A)\) such that \((x_{1}, \ldots, x_{d}) \geqslant (y_{1},
	\ldots, y_{d})\) and \(y_{i} = 0\) for all \(i \in A\). If \(g(A)\) is
	empty, let \(\text{down}(g(A))\) be the singleton containing the zero tuple.
	The set \(\text{down}(g(A))\) is finite, with an upper bound on its size:
	\[ |\text{down}(g(A))| \ \leqslant \ \sum\limits_{(x_{1}, \ldots, x_{d}) \in
	g(A)} \ \prod\limits_{i \in \{1, \ldots, d\} \setminus A} (x_{i} + 1)
	\text{.} \]
	
	Using this, define a semilinear set
	\[ Y \ = \ \bigcup_{A \subseteq \{1, \ldots, d\} \ \wedge \ A \
	\text{is good}} \left( \text{down}(g(A)) + \left( \bigcup_{i \in A}
	([1 = i], \ldots, [d = i]) \right)^{\ast} \right) \text{.} \]
	By definition of \(g(A)\), we have \(Y \subseteq X\). Our goal is to prove
	that \(Y = X\).
	
	\medskip
	
	Suppose, for the sake of contradiction, that there exists a tuple \((x_{1},
	\ldots, x_{d}) \in X \setminus Y\). We will construct a tuple \((y_{1},
	\ldots, y_{d}) \in \mathbb{N}^{d}\) such that \((x_{1}, \ldots, x_{d})
	\leqslant (y_{1}, \ldots, y_{d})\) and \((y_{1}, \ldots, y_{d}) \in X\).
	Since \(Y\) is downward closed, this will imply \((y_{1}, \ldots, y_{d}) \in
	X \setminus Y\).
	
	\medskip
	
	Proceed through indices \(i = 1, \ldots, d\), choosing \(y_{i}\) iteratively
	while maintaining a set \(A\), initially empty. Assume \(y_{1}, \ldots,
	y_{i - 1}\) have been chosen. For all \(j \in \{1, \ldots, i - 1\}\), define
	\(y_{j}' = \omega\) if \(j \in A\), and \(y_{j}' = y_{j}\) otherwise. If
	there exists a number \(z_{i} \in \mathbb{N}\) such that
	\[ (y_{1}', \ldots, y_{i - 1}', z_{i}, x_{i + 1}, \ldots, x_{d}) \ \in \ X
	\qquad \text{and} \qquad (y_{1}', \ldots, y_{i - 1}', z_{i} + 1, x_{i + 1},
	\ldots, x_{d}) \ \notin \ X \text{,} \]
	set \(y_{i} = z_{i}\). Otherwise, let \(y_{i} = x_{i}\) and add \(i\) to
	\(A\). Note that \(y_{i} \geqslant x_{i}\).
	
	\medskip
	
	By construction, \((y_{1}, \ldots, y_{d}) \in X\), and \(A\) is good.
	Furthermore,
	\[ \underset{i \in \{1, \ldots, d\} \setminus A} {\forall} \ (y_{1}',
	\ldots, y_{i - 1}', y_{i} + 1, y_{i + 1}', \ldots, y_{d}') \notin X
	\text{,} \]
	where \(y_{j}'\) is as defined earlier. Thus,
	\[ (y_{1}, \ldots, y_{d}) \ \in \ \text{down}(g(A)) + \left(
	\bigcup_{i \in A} ([1 = i], \ldots, [d = i]) \right)^{\ast} \text{.} \]
	This contradiction completes the proof that \(X\) is a semilinear set.
	
	\medskip
	
	Now we proceed to the second part of the problem. Let \(T \subseteq
	\mathbb{Z}^{d}\) be the set of available transitions. Define \(X\) as the
	downward closure of configurations reachable from \(x\).
	
	\medskip
	
	Consider a directed graph \(G\) where \(V(G) = (\mathbb{N} \cup
	\{\omega\})^{d}\) and \(E(G)\) contains pairs \(((y_{1}, \ldots, y_{d}),
	\allowbreak (z_{1}, \ldots, z_{d}))\) such that there exists a transition
	\((t_{1}, \ldots, t_{d}) \in T\) satisfying:
	\[ y_{i} = z_{i} = \omega \qquad \text{or} \qquad y_{i}, z_{i} \in
	\mathbb{N} \ \wedge \ y_{i} + t_{i} = z_{i} \text{,} \]
	for each \(i \in \{1, \ldots, d\}\). Run a breadth-first search (BFS) on
	this graph, starting at vertex \(x\).
	
	\medskip
	
	Whenever a vertex \((y_{1}, \ldots, y_{d})\) is encountered such that there
	exists an ancestor \((z_{1}, \ldots, z_{d})\) in the BFS-tree with \((z_{1},
	\ldots, z_{d}) \leqslant (y_{1}, \ldots, y_{d})\), perform the following:
	\begin{enumerate}
		\item For each \(i \in \{1, \ldots, d\}\), set \(y_{i}' = \omega\) if
		\(y_{i} > z_{i}\), or \(y_{i}' = y_{i}\) otherwise;
		\item If \((y_{1}', \ldots, y_{d}')\) has not been visited, add it to
		      the BFS queue and mark it as visited;
	\end{enumerate}
	Note that \((y_{1}', \ldots, y_{d}') \geqslant (y_{1}, \ldots, y_{d})\), so
	we do not need to process the neighbors of \((y_{1}, \ldots, y_{d})\). We do
	this because if \((y_{1}, \ldots, y_{d})\) and \((z_{1}, \ldots, z_{d})\)
	belong to \(X\), then \((y_{1}', \ldots, y_{d}')\) also belongs to \(X\).
	
	\medskip
	
	Suppose, for the sake of contradiction, that the BFS does not terminate.
	Then there exists an infinite sequence of vertices:
	\[ Y_{0} \ = \ \big( \big( y_{1}^{1}, \ldots, y_{d}^{1} \big), \big(
	y_{1}^{2}, \ldots, y_{d}^{2} \big), \ldots, \big) \]
	where each vertex \((y_{1}^{i}, \ldots, y_{d}^{i})\) is the parent of the
	next. Since \(\big( (\mathbb{N} \cup \{\omega\})^{d}, \leqslant \big)\) is a
	well-quasi-order, there exist indices \(i, j \in \mathbb{N}^{+}\) with \(i <
	j\) such that \(\big( y_{1}^{i}, \ldots, y_{d}^{i} \big) \leqslant \big(
	y_{1}^{j}, \ldots, y_{d}^{j} \big)\). Now consider the sequence:
	\[ Y_{1} \ = \ \big( \big( y_{1}^{j}, \ldots, y_{d}^{j} \big), \big(
	y_{1}^{j + 1}, \ldots, y_{d}^{j + 1} \big), \ldots, \big) \text{.} \]
	In this sequence, each tuple has at least one element equal to \(\omega\).
	Repeating this argument produces sequences \(Y_{2}, Y_{3}, \ldots\), where
	each tuple in \(Y_{k}\) contains at least \(k\) elements equal to
	\(\omega\). The existence of \(Y_{d + 1}\) leads to a contradiction.
	Therefore, the BFS terminates
	
	\medskip
	
	Let \(S\) be the set of visited vertices. The answer is:
	\[ X \ = \ \bigcup_{y \in S} \Bigg( \text{down}(\{y\}) + \Bigg( \bigcup_{i
	\in \{1, \ldots, d\} \ \wedge \ y_{i} = \omega} ([1 = i], \ldots, [d = i])
	\Bigg)^{\ast} \Bigg) \text{.} \]
	The algorithm is correct because for every \(y\) coverable by \(x\), there
	exists \(y' \in S\) such that \(y' \geqslant y\).
	
	\bigskip
	
	\textbf{Problem 4}
	
	\medskip
	
	This problem was solved in collaboration with Kacper Bal and Mateusz
	Mroczka.
	
	\medskip
	
	To begin, we construct an injection \(f : \{0, 1\}^{\ast} \to \mathbb{N}\),
	defined as:
	\[ f(s) \ = \ 2^{n} + \sum\limits_{i = 0}^{n - 1} [s[i + 1] = 1] \cdot 2^{i}
	\text{,} \]
	where \(s\) is a string of length \(n\). Note that every natural number
	except \(0\) lies in the image of this function.
	
	\medskip
	
	We will utilize the following formulas:
	\begin{align*}
		\psi_{\text{zero}}(a) \ &\coloneqq \ \underset{b \in \mathbb{N}}
		{\forall} \ a + b = b \text{,} \\
		\psi_{\text{one}}(a) \ &\coloneqq \ \underset{b \in \mathbb{N}}{\forall}
		\ a \times b = b \text{,} \\
		\psi_{\text{even}}(a) \ &\coloneqq \ \underset{b \in \mathbb{N}}
		{\exists} \ a = b + b \text{,} \\
		\psi_{\text{pow2}}(a) \ &\coloneqq \ \underset{b \in \mathbb{N}}
		{\forall} \left( \left( \neg \psi_{\text{zero}}(b) \ \wedge \
		\underset{c \in \mathbb{N}}{\exists} \ b \times c = a \right) \
		\Rightarrow \ (\psi_{\text{even}}(b) \ \vee \ \psi_{\text{one}}(b))
		\right) \text{,} \\
		\psi_{\leqslant}(a, b) \ &\coloneqq \ \underset{c \in \mathbb{N}}
		{\exists} \ a + c = b \text{,} \\
		\psi_{<}(a, b) \ &\coloneqq \ \underset{c \in \mathbb{N}}{\exists} \ (a
		+ c = b \ \wedge \ \neg \psi_{\text{zero}}(c)) \text{,} \\
		\psi_{\text{log}}(a, b) \ &\coloneqq \ \psi_{\text{pow2}}(b) \ \wedge \
		\psi_{\leqslant}(b, a) \ \wedge \ \psi_{<}(a, 2 \times b) \text{.}
	\end{align*}
	
	Next, we construct a function \(g\) that transforms a first-order logic
	sentence \(\varphi(s_{1}, \ldots, s_{m})\) over the free monoid
	\(\big(\{0, 1\}^{\ast}, \cdot, 0, 1 \big)\) into a sentence \(\psi(a_{1},
	\ldots, a_{m})\) over the arithmetic structure \((\mathbb{N}, +, \times)\).
	This transformation ensures that \(\varphi(s_{1}, \ldots, s_{m})\) is true
	if and only if \(\psi(a_{1}, \ldots, a_{m})\) is true.
	
	\medskip
	
	The function \(g\) is defined recursively as follows:
	\begin{align*}
		g \left( \underset{s \in \{0, 1\}^{\ast}}{\forall} \ \varphi'(s, s_{1}',
		\ldots, s_{m'}') \right) \ &\coloneqq \ \underset{a \in \mathbb{N}}
		{\forall} \ (\neg \psi_{\text{zero}}(a) \ \wedge \ g(\varphi'(s, s_{1}',
		\ldots, s_{m'}'))) \text{,} \\[0.2 cm]
		g \left( \underset{s \in \{0, 1\}^{\ast}}{\exists} \ \varphi'(s, s_{1}',
		\ldots, s_{m'}') \right) \ &\coloneqq \ \underset{a \in \mathbb{N}}
		{\exists} \ (\neg \psi_{\text{zero}}(a) \ \wedge \ g(\varphi'(s, s_{1}',
		\ldots, s_{m'}'))) \text{,} \\[0.2 cm]
		g(\varphi'(s_{1}', \ldots, s_{m'}') \ \wedge \ \varphi''(s_{1}'',
		\ldots, s_{m''}'')) \ &\coloneqq \ g(\varphi'(s_{1}', \ldots, s_{m'}'))
		\ \wedge \ g(\varphi''(s_{1}'', \ldots, s_{m''}'')) \text{,} \\[0.2 cm]
		g(\varphi'(s_{1}', \ldots, s_{m'}') \ \vee \ \varphi''(s_{1}'', \ldots,
		s_{m''}'')) \ &\coloneqq \ g(\varphi'(s_{1}', \ldots, s_{m'}')) \ \vee \
		g(\varphi''(s_{1}'', \ldots, s_{m''}'')) \text{,} \\[0.2 cm]
		g(\neg \varphi'(s_{1}', \ldots, s_{m'}')) \ &\coloneqq \ \neg
		g(\varphi'(s_{1}', \ldots, s_{m'}')) \text{.}
	\end{align*}
	
	The remaining case involves sentences of the form \(s_{1}' \cdot \ldots
	\cdot s_{m'}' = s_{1}'' \cdot \ldots \cdot s_{m''}''\). Without loss of
	generality, we assume these are reduced to \(s \cdot t = u\). Let \(a_{s}\)
	be defined as follows: if \(s\) is a variable, \(a_{s}\) is the
	corresponding variable; if \(s\) is a constant, \(a_{s}\) is \(f(s)\); and
	if \(s\) is an argument, \(a_{s}\) is the corresponding argument. The values
	\(a_{t}\) and \(a_{u}\) are defined analogously. Define:
	\[ g(s \cdot t = u) \ \coloneqq \ \underset{b_{t} \in \mathbb{N}}{\exists} \
	\left( \psi_{\text{log}}(a_{t}, b_{t}) \ \wedge \ b_{t} \times a_{s} + a_{t}
	= a_{u} + b_{t} \right) \text{.} \]
	The term \(b_{t}\) represents the most significant bit of \(a_{t}\).
	Ideally, we would express the equation as \(b_{t} \times a_{s} + (a_{t} -
	b_{t}) = a_{u}\), but since subtraction is not available in this framework,
	we add \(b_{t}\) to the right-hand side instead.
	
	\medskip
	
	Finally, for each argument \(a_{i}\) of \(\psi\), we impose the restriction
	\(\neg \psi_{\text{zero}}(a_{i})\).
	
	\medskip
	
	By this construction, \(\varphi(s_{1}, \ldots, s_{m})\) is true if and only
	if its transformed counterpart \(g(\varphi(s_{1}, \ldots, \allowbreak
	s_{m})) = \psi(a_{1}, \ldots, a_{m})\) is also true, thus completing the
	proof.
\end{document}
