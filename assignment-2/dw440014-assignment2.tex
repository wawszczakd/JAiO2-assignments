\documentclass[12pt]{article}
\usepackage[utf8]{inputenc}
\usepackage[T1]{fontenc}
\usepackage{indentfirst}
\usepackage{amsmath}
\usepackage{amssymb}
\usepackage{natbib}
\usepackage{graphicx}
\usepackage{float}
\usepackage[a4paper, margin = 2 cm]{geometry}
\usepackage{fancyhdr}
\usepackage{wrapfig}
\usepackage{hyperref}
\usepackage{mathtools}
\usepackage{algorithm}
\usepackage{algpseudocode}

\title{Languages, Automata and Computation II assignment 2}
\author{Dominik Wawszczak}
\date{2024-12-17}

\begin{document}
	\setlength{\parindent}{0 cm}
	
	Dominik Wawszczak \hfill Languages, Automata and Computation II
	
	Student ID Number: 440014 \hfill Assignment 2
	
	Group Number: 1
	
	\bigskip
	\hrule
	\bigskip
	
	\textbf{Problem 1}
	
	\medskip
	
	For any \(w \in \Sigma^{\ast}\), let \(P_{w}(x)\) denote the function
	obtained by feeding \(w\) into the automaton. By definition, this function
	is a polynomial in one variable \(x\). Let \(X\) be the set defined in the
	problem statement.
	
	\medskip
	
	We consider two cases:
	\begin{enumerate}
		\item There exists a word \(w \in \Sigma^{\ast}\) such that \(P_{w}(x)\)
		      is not the zero polynomial.
		      
		      In this case, let \(R(P_{w})\) represent the set of roots of
		      \(P_{w}(x)\). Since \(P_{w}(x)\) is not identically zero,
		      \(R(P_{w})\) is finite. Furthermore, \(X\) must be a subset of
		      \(R(P_{w})\); otherwise, there would exist some \(x \in X\) for
		      which \(P_{w}(x) \neq 0\), contradicting the definition of \(X\).
		      Thus, \(X\) is finite.
		
		\item For every word \(w \in \Sigma^{\ast}\), \(P_{w}(x)\) is the zero
		      polynomial.
		      
		      In this scenario, \(P_{w}(x) = 0\) holds for all \(x \in
		      \mathbb{Q}\) and every \(w \in \Sigma^{\ast}\). Consequently, \(X
		      = \mathbb{Q}\).
	\end{enumerate}
	From these cases, we conclude that \(X\) is either finite or equal to
	\(\mathbb{Q}\), completing the proof.
	
	\bigskip
	
	\textbf{Problem 2}
	
	\medskip
	
	This problem was solved in collaboration with Kacper Bal and Mateusz
	Mroczka.
	
	\medskip
	
	Let \(\mathcal{A}\) denote the automaton from the problem statement, with
	dimension \(d\) and alphabet \(\Sigma\). For any \(a \in \Sigma\), let
	\(M_{a} \in \mathbb{Q}^{d \times d}\) represent the matrix corresponding to
	the transition function for symbol \(a\).
	
	\medskip
	
	We will construct a polynomial automaton \(\mathcal{B}\) that defines a
	function \(g : \Sigma^{\ast} \to \mathbb{Q}\). The goal is for this function
	to satisfy the property that, for any word \(w \in \Sigma^{\ast}\), \(g(w) =
	f(w_{\text{lex}})\), where \(w_{\text{lex}}\) is \(w\) sorted
	lexicographically, and \(f\) is the function computed by \(\mathcal{A}\).
	
	\medskip
	
	Assume \(\Sigma = \{a_{1}, a_{2}, \ldots, a_{k}\}\), where \(a_{1}
	<_{\text{lex}} a_{2} <_{\text{lex}} \ldots <_{\text{lex}} a_{k}\). Denote
	the initial state of \(\mathcal{A}\) as \(q_{\text{ini}} \in
	\mathbb{Q}^{1 \times d}\), and the final mapping as \(q_{\text{fin}} \in
	\mathbb{Q}^{d \times 1}\). According to the definition of a weighted
	automaton, for any \(w \in \Sigma^{\ast}\) such that \(w = w_{1} w_{2}
	\ldots w_{n}\), it holds that \(f(w) = q_{\text{ini}} \cdot M_{w_{1}} \cdot
	M_{w_{2}} \cdot \ldots \cdot M_{w_{n}} \cdot q_{\text{fin}}\). Therefore, it
	should also hold that
	\[ g(w) \ = \ q_{\text{ini}} \cdot M_{a_{1}}^{\#_{a_{1}}(w)} \cdot
	M_{a_{2}}^{\#_{a_{2}}(w)} \cdot \ldots \cdot M_{a_{k}}^{\#_{a_{k}}(w)} \cdot
	q_{\text{fin}} \text{.} \]
	
	\medskip
	
	The automaton \(\mathcal{B}\) will have dimension \(d^{2} \cdot k\). The
	first \(d^{2}\) coordinates will correspond to \(M_{a_{1}}\) raised to the
	power of the number of letters \(a_{1}\) read so far, the next \(d^{2}\)
	coordinates to \(M_{a_{2}}\), and so on. In the initial state of
	\(\mathcal{B}\), we store the identity matrix \(I\) for each letter
	\(a_{i}\) in the alphabet.
	
	\medskip
	
	For any \(a \in \Sigma\), the transition matrix simulates matrix
	multiplication by \(M_{a}\) on the appropriate coordinates, while the other
	coordinates remain unchanged. Note that this requires only linear mappings.
	
	\medskip
	
	Once the values \(M_{a_{1}}^{\#_{a_{1}}(w)}, M_{a_{2}}^{\#_{a_{2}}(w)},
	\ldots, M_{a_{k}}^{\#_{a_{k}}(w)}\) have been computed and stored in the
	state, we need to calculate the final value
	\[ q_{\text{ini}} \cdot M_{a_{1}}^{\#_{a_{1}}(w)} \cdot
	M_{a_{2}}^{\#_{a_{2}}(w)} \cdot \ldots \cdot M_{a_{k}}^{\#_{a_{k}}(w)} \cdot
	q_{\text{fin}} \text{.} \]
	This is a polynomial involving \(d^{2} \cdot k\) variables, which are the
	elements of the matrices stored in the state. Hence, the automaton
	\(\mathcal{B}\) can compute this, making it the only non-linear transition.
	
	\medskip
	
	Since \(\mathcal{A}\) is also a polynomial automaton (as linear functions
	are polynomials), we can use the algorithm presented in the lecture to
	determine whether \(\mathcal{A}\) and \(\mathcal{B}\) are equivalent. If
	they are, the function \(f\) is commutative; otherwise, it is not, which
	concludes the proof.
	
	\bigskip
	
	\textbf{Problem 3}
	
	\medskip
	
	This problem was solved in collaboration with Kacper Bal and Mateusz
	Mroczka.
	
	\medskip
	
	Since weighted automata are a special case of polynomial automata, we only
	need to show that if a polynomial automaton \(\mathcal{A}\) computes a
	function \(f : \Sigma^{\ast} \to \mathbb{F}\), then there exists a weighted
	automaton \(\mathcal{B}\) that computes the same function.
	
	\medskip
	
	Let \(d\) be the dimension of \(\mathcal{A}\). We set the dimension of
	\(\mathcal{B}\) to \(D = |\mathbb{F}|^{d}\), which represents the number of
	possible states \(\mathcal{A}\) can be in. The invariant for \(\mathcal{B}\)
	is that it will always have exactly one coordinate equal to \(1\), with the
	rest set to \(0\). This coordinate corresponds to the state of
	\(\mathcal{A}\). In the initial state of \(\mathcal{B}\), represented as a
	vector in \(\mathbb{F}^{1 \times D}\), the coordinate corresponding to the
	initial state of \(\mathcal{A}\) is set to \(1\), with all others set to
	\(0\).
	
	\medskip
	
	For any \(a \in \Sigma\), let \(M_{a} \in
	\mathbb{F}[x_{1}, x_{2}, \ldots, x_{d}]^{d}\) be the transition polynomials
	for the letter \(a\). Since for every \((x_{1}, x_{2}, \ldots, x_{d}) \in
	\mathbb{F}^{d}\), the result of applying the transition \(M_{a}\) to
	\((x_{1}, x_{2}, \ldots, x_{d})\) is uniquely determined, we define the
	transition function of \(\mathcal{B}\) as a \(D \times D\) matrix. The entry
	at position \((i, j)\) is \(1\) if and only if applying \(M_{a}\) to
	\((x_{1}, x_{2}, \ldots, x_{d})\) results in \((y_{1}, y_{2}, \ldots,
	y_{d})\), where \(i\) corresponds to \((x_{1}, x_{2}, \ldots, x_{d})\) and
	\(j\) corresponds to \((y_{1}, y_{2}, \ldots, y_{d})\); otherwise, the entry
	is \(0\). This construction maintains the invariant of \(\mathcal{B}\) and
	ensures correct transitions.
	
	\medskip
	
	The final mapping of \(\mathcal{B}\) is a vector in
	\(\mathbb{F}^{D \times 1}\). The entry at position \(i\) is \((x_{1}, x_{2},
	\ldots, x_{d}) \cdot q_{\text{fin}}\), where \(i\) corresponds to \((x_{1},
	x_{2}, \ldots, x_{d})\), and \(q_{\text{fin}}\) is the final mapping of
	\(\mathcal{A}\). With this construction, automata \(\mathcal{A}\) and
	\(\mathcal{B}\) compute the same function, thus completing the proof.
\end{document}
