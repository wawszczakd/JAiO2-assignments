\documentclass[12pt]{article}
\usepackage[utf8]{inputenc}
\usepackage[T1]{fontenc}
\usepackage{indentfirst}
\usepackage{amsmath}
\usepackage{amssymb}
\usepackage{natbib}
\usepackage{graphicx}
\usepackage{float}
\usepackage[a4paper, margin = 2 cm]{geometry}
\usepackage{fancyhdr}
\usepackage{wrapfig}
\usepackage{hyperref}
\usepackage{mathtools}
\usepackage{algorithm}
\usepackage{algpseudocode}

\title{Languages, Automata and Computation II assignment 2}
\author{Dominik Wawszczak}
\date{2024-12-17}

\begin{document}
	\setlength{\parindent}{0 cm}
	
	Dominik Wawszczak \hfill Languages, Automata and Computation II
	
	Student ID Number: 440014 \hfill Assignment 2
	
	Group Number: 1
	
	\bigskip
	\hrule
	\bigskip
	
	\textbf{Problem 1}
	
	\medskip
	
	For any \(w \in \Sigma^{\ast}\), let \(P_{w}(x)\) denote the polynomial
	obtained by feeding \(w\) into the automaton. By definition, this function
	is a polynomial in one variable \(x\). Let \(X\) be the set defined in the
	problem statement.
	
	\medskip
	
	We consider two cases:
	\begin{enumerate}
		\item There exists a word \(w \in \Sigma^{\ast}\) such that \(P_{w}(x)\)
		      is not the zero polynomial.
		      
		      In this case, let \(R(P_{w})\) represent the set of roots of
		      \(P_{w}(x)\). Since \(P_{w}(x)\) is not identically zero,
		      \(R(P_{w})\) is finite. Furthermore, \(X\) must be a subset of
		      \(R(P_{w})\); otherwise, there would exist some \(x \in X\) for
		      which \(P_{w}(x) \neq 0\), contradicting the definition of \(X\).
		      Thus, \(X\) is finite.
		
		\item For every word \(w \in \Sigma^{\ast}\), \(P_{w}(x)\) is the zero
		      polynomial.
		      
		      In this scenario, \(P_{w}(x) = 0\) holds for all \(x \in
		      \mathbb{Q}\) and every \(w \in \Sigma^{\ast}\). Consequently, \(X
		      = \mathbb{Q}\).
	\end{enumerate}
	From these cases, we conclude that \(X\) is either finite or equal to
	\(\mathbb{Q}\), completing the proof.
\end{document}
