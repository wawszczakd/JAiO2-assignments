\documentclass[12pt]{article}
\usepackage[utf8]{inputenc}
\usepackage[T1]{fontenc}
\usepackage{indentfirst}
\usepackage{amsmath}
\usepackage{amssymb}
\usepackage{natbib}
\usepackage{graphicx}
\usepackage{float}
\usepackage[a4paper, margin = 2 cm]{geometry}
\usepackage{fancyhdr}
\usepackage{wrapfig}
\usepackage{hyperref}
\usepackage{mathtools}
\usepackage{algorithm}
\usepackage{algpseudocode}

\title{Languages, Automata and Computation II assignment 3}
\author{Dominik Wawszczak}
\date{2025-01-31}

\begin{document}
	\setlength{\parindent}{0 cm}
	
	Dominik Wawszczak \hfill Languages, Automata and Computation II
	
	Student ID Number: 440014 \hfill Assignment 3
	
	Group Number: 1
	
	\bigskip
	\hrule
	\bigskip
	
	\textbf{Problem 1}
	
	\medskip
	
	This problem was solved in collaboration with Kacper Bal and Mateusz
	Mroczka.
	
	\medskip
	
	We say that a language \(L \in \mathbb{A}^{*}\) is a \textit{good}, if it
	satisfies the condition from the statement, i.e.
	\[ \underset{w \in \mathbb{A}^{\ast}}{\forall} \
	\underset{\sigma : \mathbb{A} \to \mathbb{A}}{\forall} \ (w \in L \ \iff \
	\sigma(w) \in L) \text{.} \]
	
	\medskip
	
	Let \(A \in \mathbb{A}\) be an arbitrary element. Define \(\sigma_{A}\) as
	the function constantly equal to \(A\). For any good language \(L \in
	\mathbb{A}^{\ast}\), it must hold that
	\[ \underset{w \in \mathbb{A}^{\ast}}{\forall} \ (w \in L \ \iff \
	\sigma_{A}(w) \in L) \text{,} \]
	since the order of quantifiers does not matter. This is equivalent to
	\[ \underset{w \in \mathbb{A}^{\ast}}{\forall} \ \big( w \in L \ \iff \
	A^{|w|} \in L \big) \text{,} \]
	which implies that for every \(n \in \mathbb{N}\), \(L\) either contains all
	words of length \(n\) or none of them.
	
	\medskip
	
	The problem of determining whether a language \(L \in \mathbb{A}^{\ast}\) is
	good is semi-decidable, since we can iterate through register automata that
	use zero registers, comparing each one with the automaton from the input.
	
	\medskip
	
	The complement of this problem is also semi-decidable, as we can iterate
	over words of length \(n\), for \(n \in \mathbb{N}\), considering only a
	subset of \(n\) symbols from \(\mathbb{A}\). For each such word, we can
	check all the possible functions \(\sigma\) limited to this subset and
	verify whether the condition holds.
	
	\medskip
	
	Since both the problem and its complement are semi-decidable, the problem is
	decidable, which completes the proof.
\end{document}
